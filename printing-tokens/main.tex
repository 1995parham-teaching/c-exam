\قسمت{چاپ کردن توکن‌ها}

رشته‌ی \متن‌لاتین{s} به شما داده شده است، تابعی بنویسید که کلمات این رشته، که با کاراکتر فاصله از یکدیگر جدا شده‌اند، را چاپ کند.
هر کلمه باید در یک خط چاپ شود.
اندازه رشته و کلمات آن محدودیتی \متن‌سیاه{ندارند}.

\begin{latin}
\begin{minted}[]{c}
void printing_tokens(char *s) {
  // write your code here
}
\end{minted}
\end{latin}

به طور مثال فرض کنید:

\begin{latin}
\begin{minted}[]{c}
char *s = "This is C"
\end{minted}
\end{latin}

در این صورت خروجی تابع شما باید به شکل زیر باشد:

\begin{latin}
\begin{minted}[]{output}
This
is
C
\end{minted}
\end{latin}


۴ نمره

\شروع{پاسخ}

دو روش کلی برای این سوال وجود دارد. در روش اول رشته را کاراکتر به کاراکتر خوانده و آن‌ها را چاپ می‌کنیم و به محض پیدا کردن فاصله به جای آن کاراکتر \متن‌لاتین{newline} را چاپ می‌کنیم.
در روش دوم با استفاده از تابع \متن‌لاتین{strchr} اولین محل فاصله را پیدا می‌کنیم، در این روش نیاز داریم که زیر رشته از ابتدای رشته تا به اینجا را نگهداری کنیم و از همین رو، با استفاده از تابع \متن‌لاتین{malloc}
یک زیر رشته جدید ساخته و از ابتدای رشته به اینجا را با استفاده از تابع \متن‌لاتین{strcpy} در آن ذخیره می‌کنیم. در ادامه نیاز داریم اشاره‌گر به ابتدای رشته را تغییر دهیم تا در فراخوانی بعدی \متن‌لاتین{strchr}
محل رخداد فاصله بعدی را پیدا کنیم.

\begin{latin}
  \inputminted{c}{./printing-tokens/main.c}
\end{latin}

\پایان{پاسخ}
