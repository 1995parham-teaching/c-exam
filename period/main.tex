\قسمت{نقطه گذاری}
تابع زیر نام دو فایل را دریافت می‌کند.
این تابع هر خط از فایل اول را خوانده و یک نقطه
\متن‌لاتین{\متن‌ایتالیک{'.'}}
به انتهای آن اضافه کرده و آن را در فایل دوم یادداشت می‌کند.

\begin{latin}
\begin{minted}{c}
void add_period(char* f1, char* f2) {
    // your code goes here
}
\end{minted}
\end{latin}

برای درک بهتر از عملکرد تابع به مثال‌های زیر توجه کنید:

\begin{latin}
\begin{minted}{output}
Input:
    Hello world
    123
Output:
    Hello world.
    123.

Input:
    123123
Output:
    123123.
\end{minted}
\end{latin}

\شروع{شمارش}[ا -]

\فقره فرض کنید طول هر خط حداکثر برابر با ۲۰۰ کاراکتر باشد. ۴ نمره

\فقره فرض کنید طول خط‌ها از پیش مشخص نشده است. ۶ نمره

\پایان{شمارش}

در نظر داشته باشید که حتما در هنگام باز کردن فایل خطا را نیز مورد بررسی قرار دهید.
مشخص است که تنها نیاز به حل یکی از قسمت‌ها می‌باشد و حل هر دو قسمت نمره‌ی اضافه‌ای ندارد.

\شروع{پاسخ}

\begin{latin}
  \inputminted{c}{./period/main.c}
\end{latin}

\پایان{پاسخ}
