\documentclass[../main.tex]{subfiles}

\begin{document}
\section{نقطه گذاری}
\paragraph{}
تابع زیر نام دو فایل را دریافت می‌کند.
این تابع هر خط از فایل اول را خوانده و یک نقطه
\lr{\textit{'.'}}
به انتهای آن اضافه کرده و آن را در فایل دوم یادداشت می‌کند.

\begin{latin}
\begin{lstlisting}[language=c]
    void add_period(char* f1, char* f2) {
        // your code goes here
    }
\end{lstlisting}
\end{latin}

برای درک بهتر از عملکرد تابع به مثال‌های زیر توجه کنید:

\begin{latin}
\begin{verbatim}
    Input:
        Hello world
        123
    Output:
        Hello world.
        123.
\end{verbatim}

\begin{verbatim}
    Input:
        123123
    Output:
        123123.
\end{verbatim}
\end{latin}

\paragraph{قسمت اول}
فرض کنید طول هر خط حداکثر برابر با ۲۰۰ کاراکتر می‌باشد.

\paragraph{}
۴ نمره

\paragraph{قسمت دوم}
فرض کنید طول خط‌ها از پیش مشخص نشده است.

\paragraph{}
۶ نمره

\paragraph{نکات}
در نظر داشته باشید که حتما در هنگام باز کردن فایل خطا را نیز مورد بررسی قرار دهید.
مشخص است که در تنها نیاز به حل یکی از قسمت‌ها می‌باشد و حل هر دو قسمت نمره‌ی اضافه‌ای ندارد.

\end{document}
