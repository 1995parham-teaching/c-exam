\documentclass[../main.tex]{subfiles}

\begin{document}
\section{آیا آناگرام}
\paragraph{}
دو رشته آناگرام می‌باشند اگر حروف آن‌ها تنها جایگشتی از یکدیگر باشد.
تابع زیر دو رشته که از حروف کوچک انگلیسی تشکیل شده‌اند را دریافت کرده و مشخص می‌کند
آیا این دو رشته آناگرام می‌باشند یا خیر.
در صورتی که دو رشته آناگرام باشند خروجی این تابع برابر با یک و در غیر این صورت برابر با صفر می‌باشد.

\begin{latin}
\begin{lstlisting}[language=c]
    int is_anagram(char* s1, char* s2) {
        // your code goes here
    }
\end{lstlisting}
\end{latin}

برای درک بهتر از عملکرد تابع به مثال‌های زیر توجه کنید:

\begin{latin}
\begin{verbatim}
    Input: eat tea
    Output: 1
\end{verbatim}

\begin{verbatim}
    Input: dad day
    Output: 0
\end{verbatim}

\begin{verbatim}
    Input: bob boo
    Output: 0
\end{verbatim}

\begin{verbatim}
    Input: eat ate
    Output: 1
\end{verbatim}
\end{latin}

\paragraph{راهنمایی}
تعداد حروف کوچک انگلیسی برابر با ۲۶ می‌باشد.

\paragraph{}
۴ نمره

\end{document}