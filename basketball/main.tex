\documentclass[../main.tex]{subfiles}

\begin{document}

\قسمت{شبیه‌سازی مسابقات بسکتبال}

\زیرقسمت{مقدمه}

در این پروژه قصد داریم شبیه‌سازی را برای مسابقات بسکتبال پیاده‌سازی کنیم.

\زیرقسمت{در مورد شبیه‌ساز}

روش کار این شبیه‌ساز به این صورت است که تعدادی تیم در مسابقه وجود دارند (تیم‌ها ویژگی خاصی ندارد و با ترتیب ورودشان از هم جدا می‌شوند)
و با یکدیگر بازی می‌کنند و به صورت تصادفی در هر بازی برنده مشخص می‌شود و تعدادی کاربر برد این تیم‌ها را پیش‌بینی می‌کنند.
در صورتی که تیمی که کاربر آن را قهرمان پیش‌بینی کرده است، \متن‌سیاه{قهرمان} شود، وی برنده می‌شود.

کاربر تیم‌های داخل بازی را یکی یکی در شبیه‌ساز اضافه می‌کند
سپس کاربر، پیش‌بینی افراد متفاوتی را با \متن‌سیاه{کدملی}های آن افراد در ورودی وارد می‌کند.
هر فرد (که با کدملی شناخته می‌شود) می‌تواند حداکثر 2 پیش‌بینی انجام دهد،
یعنی بگوید تیم \متن‌لاتین{X} قهرمان می‌شود یا تیم \متن‌لاتین{Y}.

در نهایت بعد از وارد شدن اطلاعات تیم‌ها و افراد پیش‌بینی کننده، با درخواست کاربر، شبیه‌سازی انجام می‌شود و
بازی‌ها بین دو به دوی تیم‌ها برگزار می‌شود.
این بازی‌ها به صورت
\متن‌سیاه{امتیازی و یک طرفه}
بوده و احتمال برد هر تیم‌، یکسان و برابر با
\(\frac{1}{2}\)
است. 
(دقت کنید که حالت مساوی نداریم و حتما یک تیم برنده می‌شود و یک تیم بازنده)
امتیاز تیم برنده برابر با ۱ و تیم بازنده برابر با صفر خواهد بود.
روند بازی‌ها برای چهار تیم با شماره‌های ۰ تا ۳ به شکل زیر خواهد بود:

\begin{latin}
\begin{lstlisting}[]
    0 plays with 1
    0 plays with 2
    0 plays with 3

    1 plays with 2
    1 plays with 3

    2 plays with 3
\end{lstlisting}
\end{latin}

\زیرقسمت{پیاده‌سازی}

در ادامه روند اجرای شبیه‌ساز را مرور می‌کنیم:

\begin{latin}
\begin{lstlisting}[]
    Welcome to our simulator
    [n_participants: 0] [n_teams: 0]

    1) Add Team
    2) Vote for a Winner
    3) Execute simulation and report a winner
\end{lstlisting}
\end{latin}

در ابتدا منو فوق نمایش داده می‌شود
که در آن از کاربر خواسته می‌شود از بین عملیات‌های زیر یک عملیات را انتخاب کند:

\شروع{شمارش}
\فقره \متن‌سیاه{عملیات افزودن تیم}: یک تیم به شمار تیم‌های موجود در شبیه‌سازی می‌افزاید.
    برای تیم لازم نیست هیچ اطلاعات از کاربر گرفته‌شود. این عملیات خروجی خاصی ندارد.

\فقره \متن‌سیاه{عملیات پیش‌بینی یک کاربر برای برد یک تیم}:
    کاربر برنامه کدملی یک شخص را وارد می‌کند و در ادامه یک شماره یک تیم را وارد می‌کند
    که در واقع فردی که کد ملی‌اش وارد شده پیش‌بینی می‌کند آن تیم برنده می‌شود.
    یک کد ملی حداکثر می‌تواند دو پیش‌بینی داشته باشد
    یعنی حداکثر ۲ بار می‌تواند از گزینه‌ی
    \متن‌لاتین{Vote for a Winner}
    استفاده کند.

\فقره با انتخاب این گزینه شبیه‌ساز اجرا شده و به صورت تصادفی تیم‌ها برنده یا بازنده می‌شود
    و در نهایت برنده لیگ گزارش می‌شود.
    همچنین شبیه‌ساز اعلام می‌کند چه کد ملی‌هایی پیش‌بینی صحیح داشتند و در نهایت برنامه خاتمه پیدا می‌کند.
\پایان{شمارش}

\متن‌سیاه{برنامه شما باید بررسی کند کد ملی یک عدد حداکثر ۱۰ رقمی بوده و شماره تیم نیز در بازه تعداد تیم‌ها باشد}.
بعد از انجام گزینه ۱ یا ۲ دوباره منو ابتدایی نمایش داده می‌شود.

\زیرزیرقسمت{نمونه ورودی و خروجی}

فرض کنید در اولین گام، کاربر اصلی ۴ تیم را وارد کنید یعنی از گزینه ۱ چهار بار استفاده کند:

\begin{latin}
\begin{lstlisting}[]
    Welcome to our simulator
    [n_participants: 0] [n_teams: 1]

    1) Add Team
    2) Vote for a Winner
    3) Execute simulation and report a winner
\end{lstlisting}
\end{latin}


\begin{latin}
\begin{lstlisting}[]
    Welcome to our simulator
    [n_participants: 0] [n_teams: 2]

    1) Add Team
    2) Vote for a Winner
    3) Execute simulation and report a winner
\end{lstlisting}
\end{latin}

\begin{latin}
\begin{lstlisting}[]
    Welcome to our simulator
    [n_participants: 0] [n_teams: 3]

    1) Add Team
    2) Vote for a Winner
    3) Execute simulation and report a winner
\end{lstlisting}
\end{latin}

\begin{latin}
\begin{lstlisting}[]
    Welcome to our simulator
    [n_participants: 0] [n_teams: 4]

    1) Add Team
    2) Vote for a Winner
    3) Execute simulation and report a winner
\end{lstlisting}
\end{latin}

در ادامه فرض کنید به اشتباه عدد ۵ (که در گزینه‌ها موجود نیست) را وارد کند:

\begin{latin}
\begin{lstlisting}[]
    Invalid Choice
    Welcome to our simulator
    [n_participants: 0] [n_teams: 4]

    1) Add Team
    2) Vote for a Winner
    3) Execute simulation and report a winner
\end{lstlisting}
\end{latin}

در ادامه می‌خواهیم پیش‌بینی تیم برنده را انجام دهیم، برای اینکار گزینه‌ی ۲ را وارد می‌کنیم:

\begin{latin}
\begin{lstlisting}[]
    ID: 0017784646
    Bet: 1
    insertion success
    Welcome to our simulator
    [n_participants: 1] [n_teams: 4]

    1) Add Team
    2) Vote for a Winner
    3) Execute simulation and report a winner
\end{lstlisting}
\end{latin}

در ادامه همین شرکت‌کننده می‌خواهد روی تیم شماره صفر نیز پیش‌بینی انجام دهد و به این ترتیب داریم:

\begin{latin}
\begin{lstlisting}[]
    ID: 0017784646
    Bet: 0
    insertion success
    Welcome to our simulator
    [n_participants: 2] [n_teams: 4]

    1) Add Team
    2) Vote for a Winner
    3) Execute simulation and report a winner
\end{lstlisting}
\end{latin}

حال ایشان به اشتباه می‌خواهد سومین پیش‌بینی خود را نیز ثبت کند که با خطا مواجه می‌شود:

\begin{latin}
\begin{lstlisting}[]
    ID: 0017784646
    Bet: 2
    insertion failed
    Welcome to our simulator
    [n_participants: 2] [n_teams: 4]

    1) Add Team
    2) Vote for a Winner
    3) Execute simulation and report a winne
\end{lstlisting}
\end{latin}

در ادامه یک شرکت‌کننده دیگر می‌خواهد روی تیم‌های شماره‌ی ۲ و ۳ پیش‌بینی انجام دهد:

\begin{latin}
\begin{lstlisting}[]
    1) Add Team
    2) Vote for a Winner
    3) Execute simulation and report a winner
    2
    ID: 0022552898
    Bet: 2
    insertion success
    Welcome to our simulator
    [n_participants: 3] [n_teams: 4]

    1) Add Team
    2) Vote for a Winner
    3) Execute simulation and report a winner
    2
    ID: 0022552898
    Bet: 3
    insertion success
    Welcome to our simulator
    [n_participants: 4] [n_teams: 4]

    1) Add Team
    2) Vote for a Winner
    3) Execute simulation and report a winner
\end{lstlisting}
\end{latin}

در نهایت با استفاده از گزینه ۳ شبیه‌سازی را اجرا می‌کنیم.
با انتخاب این گزینه شبیه‌ساز اجرا شده و به صورت تصادفی تیم‌ها برنده یا بازنده می‌شود
و در نهایت برنده لیگ گزارش می‌شود. همچنین شبیه‌ساز اعلام می‌کند چه کد ملی‌هایی پیش‌بینی صحیح داشتند
و در نهایت برنامه خاتمه پیدا می‌کند.

\begin{latin}
\begin{lstlisting}[]
    team 2 is a winner
    teams[0]: 0
    teams[1]: 2
    teams[2]: 3
    teams[3]: 1
    0022552898 win
\end{lstlisting}
\end{latin}

و برنامه به پایان می‌رسد. در پایان تیم برنده اعلام شده و امتیاز همه تیم‌ها لیست می‌شود. در صورتی که چند تیم امتیاز برابر داشته باشند به دلخواه یکی از آن‌ها برنده اعلام می‌شود.

\زیرقسمت{نکات پیاده‌سازی}

\شروع{فقرات}
    \فقره تا جایی که می‌توانید برنامه‌ی خود را به توابع معنی‌دار بشکنید. بهتر است هر تابع یک کار معین را انجام دهد.
    \فقره تعداد شرکت‌کننده‌ها و تیم‌ها هر کدام حداکثر ۱۰۰ می‌باشد و تضمین می‌شود بیشتر از این تعداد نخواهند بود.
    \فقره دقت کنید که کد ملیَ برنده‌ها باید به شکل صحیح (با صفرهای ابتدایی) چاپ شود. برای این موضوع در نظر داشته باشید که کد ملی با صفرهای ابتدایی دقیقا ۱۰ رقم است.
    \فقره برای نگهداری کد ملی از متغیر صحیح استفاده کنید و در نظر داشته باشید که کد ملی از نظر مقدار عددی بزرگ می‌باشد.
    \فقره برای پیاده‌سازی این پروژه تنها نیاز به آنچه \متن‌سیاه{تا انتهای توابع} بحث شده است دارید اما استفاده از تمام ویژگی‌های زبان سی آزاد می‌باشد.
    \فقره در پیاده‌سازی این پروژه فقط از \متن‌سیاه{زبان سی} استفاده کنید. در صورتی که برنامه شما برای اجرا نیاز به تنظیمات خاصی دارد یک راهنما برای اجرای برنامه بنویسید در فایل زیپتان همراه برنامه ارسال کنید.
    \فقره از آنجایی که این پروژه در قالب \متن‌سیاه{امتحان میانترم} می‌باشد از تغییر دادن صورت مساله یا انجام موارد خارج از موارد مطرح شده خودداری کنید.
\پایان{فقرات}

\زیرقسمت{موارد امتیازی}

\شروع{شمارش}
    \فقره شرکت‌کننده بتواند از پیش‌بینی انصراف دهد. انصراف به این ترتیب خواهد بود که تمامی پیش‌بینی‌های ایشان از لیست پیش‌بینی‌ها حذف خواهد شد. برای اینکار گزینه جدیدی مثلا 4 را در نظر بگیرید.
    \فقره چاپ جدول بازی‌ها: در انتهای بازی، همراه لیست برنده‌ها، جدول بازی‌ها و به همراه نتیجه‌شان چاپ شود.
    \فقره بازی‌ها به صورت حذفی باشند. به این ترتیب که تیم بازنده از دور بازی‌ها خارج شده و دیگر در بازی‌ها شرکت نمی‌کند.
\پایان{شمارش}

\end{document}
