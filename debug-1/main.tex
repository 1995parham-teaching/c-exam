\documentclass[../main.tex]{subfiles}

\begin{document}
\section{خطایابی ۱}
\paragraph{}
محمد قصد دارد تابعی بنویسد که با دریافت دو مجموعه اجتماع آن‌ها را محاسبه کند.
برای اینکار تابع محمد دو آرایه را ورودی گرفته است
و یک آرایه نیز برای نوشتن خروجی از کاربر گرفته شده است.

\begin{latin}
\begin{lstlisting}[language=c]
    int union(int* a, int* b, int* r) {
        int a_size = sizeof(a) / sizeof(int);
        int b_size = sizeof(b) / sizeof(int);

        int r_size = 0;

        for (int i = 0; i < b_size; i++) {
            r[r_size++] = b[i];
        }

        for (int i = 0; i < a_size; i++) {
            int flag = 0;
            for (int j = 0; j < b_size; j++) {
                if (a[i] == b[j]) {
                    flag = 1;
                    break;
                }
            }
            if (flag == 0) {
                r[r_size++] = a[i];
            }
        }

        return r_size;
    }
\end{lstlisting}
\end{latin}

\paragraph{سوال اول}
چرا خروجی تابع محمد به شکل \lr{int}
می‌باشد؟ اگر خروجی به شکل \lr{void} باشد چه مشکلی برای تابع فراخوانی کننده به وجود می‌آید؟

\paragraph{}
۲ نمره

\paragraph{سوال دوم}
خطای کد را توضیح داده و آن را اصلاح کنید.

\paragraph{}
۳ نمره

\paragraph{سوال سوم}
روش محمد برای محاسبه اجتماع را در یک خط توضیح دهید.

\paragraph{}
۲ نمره

\end{document}