
\documentclass[../main.tex]{subfiles}

\begin{document}

\قسمت{کوکاکولا}

سامان می‌خواهد امسال میز مهربانی نوشابه راه‌اندازی کند. کار به این شکل است، سامان هر روز با یک تعداد نوشابه میز را افتتاح می‌کند.
در ادامه یک صف تشکیل می‌شود که در آن هر فرد یا تعدادی نوشابه برداشته یا تعدادی نوشابه را روی میز می‌گذارد.
اگر فردی بخواهد $n$ نوشابه بردارد اما روی میز به این تعداد نوشابه وجود نداشته باشد آن فرد قهر می‌کند.
برنامه‌ای بنویسید که با دریافت افراد داخل صف و موجودی اول میز، مشخص کند دست آخر چند نوشابه روی میز می‌مانند و چند نفر قهر کرده به خانه برمی‌گردند.

برنامه شما می‌بایست برای هر فرد داخل صف یک خط از ورودی بخواند، این خط شامل یک علامت + یا -‌ است که نشان می‌دهد این فرد می‌خواهد نوشابه‌هایش را اهدا کند یا می‌خواهد تعدادی نوشابه را از میز بردارد.

\begin{latin}
\begin{minted}[]{output}
Input:
    5 7
    + 5
    - 10
    - 20
    + 40
    - 20
Output:
    22 1
\end{minted}
\end{latin}

در مثال فوق سامان کار خود را با صف پنج نفری و ۷ نوشابه آغاز می‌کند.

\شروع{فقرات}
\فقره اولین نفر ۵ نوشابه می‌دهد پس او ۱۲ نوشابه خواهد داشت.
\فقره دومین نفر ۱۰ نوشابه را برداشته پس ۲ نوشابه باقی می‌ماند.
\فقره سومین نفر می‌خواهد ۲۰ نوشابه بردارد اما این تعداد نوشابه روی میز قرار ندارد پس هیچ نوشابه‌ای برنداشته و قهر می‌کند.
\فقره چهارمین نفر ۴۰ نوشابه را به میز اضافه می‌کند پس ۴۲ نوشابه باقی می‌ماند.
\فقره پنجمین نفر ۲۰ نوشابه را برداشته و ۲۲ نوشابه باقی‌ماند. در نهایت ۱ نفر قهر کرده است و ۲۲ نوشابه باقی مانده است.
\پایان{فقرات}

\end{document}
