\documentclass[../main.tex]{subfiles}

\begin{document}
\section{عدد برتر}
\paragraph{قسمت اول}
قطعه کدی بنویسید که آرایه‌ای از اعداد صحیح را با استفاده از تابع
\lr{qsort}
مرتب کند.

\paragraph{قسمت دوم}
عدد برتر در یک آرایه با اندازه \lr{n}
عددی است که بیش از \lr{\(\frac{n}{2}\)}
بار تکرار شده است.
تابع زیر را در نظر بگیرید که در آن یک آرایه از اعداد صحیح به شما داده شده است. با فرض اینکه این آرایه حتما دارای عدد برتر است تابع را به شکلی تکمیل کنید که
این عدد را پیدا کرده و آن را بازگرداند.
در نظر داشته باشید که از قسمت اول می‌بایست در حل این قسمت استفاده کنید.

\begin{latin}
\begin{lstlisting}[language=c]
    int majority_element(int *nums, int size) {
        // your code goes here
    }
\end{lstlisting}
\end{latin}

برای درک بهتر از عملکرد تابع به مثال‌های زیر توجه کنید:

\begin{latin}
\begin{verbatim}
    Input: [3,2,3]
    Output: 3
\end{verbatim}

\begin{verbatim}
    Input: [2,2,1,1,1,2,2]
    Output: 2
\end{verbatim}
\end{latin}

\paragraph{}
۴ نمره

\end{document}