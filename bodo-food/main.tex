\documentclass[../main.tex]{subfiles}

\begin{document}
\قسمت{بُدوفود}

\زیرقسمت{مقدمه}
\پاراگراف{}
قصد داریم یک استارآپ برای تحویل غذا از رستوران‌ها به مشتریان ایجاد کنیم. در این سامانه ما تنها پیک‌ها را مدیریت می‌کنیم و مدیر سامانه می‌تواند پیک‌ها و سفارشات را از طریق پنلی که در اختیارش قرار می‌دهیم، مدیریت کند.
بنابراین اینجا کاری با رستوران‌ها و مشتریان نخواهیم داشت.

\زیرقسمت{پیک‌ها}
\پاراگراف{}
مدیر سامانه در جهت مدیریت پیک‌ها می‌تواند اعمال زیر را انجام دهد:
\شروع{شمارش}
\فقره اطلاعات یک پیک جدید را ثبت نماید.
\فقره پیک را فعال یا غیرفعال نماید.
\فقره اطلاعات یک پیک را حذف نماید.
\فقره یک پیک را به صورت تصادفی برای یک سفارش برگزیند.
\پایان{شمارش}

\پاراگراف{}
شما می‌بایست برای نگهداری اطلاعات پیک‌ها از آرایه‌های یک بعدی استفاده کنید و اندازه این آرایه‌ها را ثابت در نظر بگیرید (مثلا ۱۰۰). از آنجایی که هنوز راهی برای نگهداری چند فیلد به صورت همزمان در یک ساختمان داده‌ای نخوانده‌ایم، شما می‌بایست برای هر یک از فیلدهای پیک از یک آرایه جداگانه استفاده کنید (این مورد را بیشتر توضیح خواهیم داد).
برای هر پیک به اطلاعات زیر نیاز داریم:
\شروع{فقرات}
\فقره کد ملی - عدد صحیح - حداکثر ۱۰ رقمی (بدون احتساب صفرهای اول)
\فقره شماره تماس - عدد صحیح - دقیقا ۱۰ رقمی (بدون احتساب صفر اول)
\پایان{فقرات}

\پاراگراف{}
شما می‌بایست این اطلاعات را از کاربر گرفته و بعد از اعمال صحت‌سنجی دست به نگهداری آن‌ها بزنید. برای نگهداری این اطلاعات همانطور که گفته شد نیاز به استفاده همزمان از چند آرایه دارید. یعنی:
\begin{latin}
\begin{lstlisting}[language=c]
#define MAX 100

int identities[MAX];
int phones[MAX];
int status[MAX];
\end{lstlisting}
\end{latin}

\پاراگراف{}
در نظر داشته باشید که شما می‌بایست در زمان ذخیره پیک $i$ام، شماره تماس ایشان را در آرایه مربوطه و در خانه $i$ام و به همین ترتیب کد ملی ایشان را در آرایه مربوطه و در خانه $i$ام نگهداری کنید و به همین ترتیب به این اطلاعات دسترسی داشته باشید.

\پاراگراف{}
از آنجایی که طول آرایه ثابت است ما حداکثر تعداد مشخصی از پیک‌ها را می‌توانیم پشتیبانی کنیم اما در نظر داشته باشید که امکان حذف یک پیک نیز وجود دارد، پس شما می‌بایست به نحوی این آرایه را مدیریت کنید که بعد از حذف یک پیک بتوان از خانه‌ی آن برای نگهداری اطلاعات سایر پیک‌ها استفاده کرد.
برای این کار به عنوان مثال می‌توانید همه خانه‌های بعد از آن را یک واحد به سمت چپ حرکت بدهید تا آن خانه خالی، پر شده و یک خانه خالی در انتهای آرایه ایجاد شود.

\زیرقسمت{سفارشات}

\پاراگراف{}
مدیر سامانه می‌تواند در سامانه آغاز و پایان یک سفارش را ثبت کند. در زمان ثبت سفارش می‌بایست یک پیک تصادفی از میان پیک‌های فعال به سفارش تخصیص پیدا کند. در نظر داشته باشید این پیک انتخابی تا زمانی که کاربر دستور پایان سفارش را نزده، غیرفعال خواهد بود.
هر سفارش یک شناسه‌ی تصادفی دارد که می‌بایست هنگام شروع سفارش ساخته شده و به آن تخصیص یابد و به کاربر نیز نمایش داده شود. سفارش‌ها می‌بایست تا زمان پایان، نگهداری شوند و پس از پایان سفارش، پیکِ سفارش به حالت فعال برگشته و سفارش از لیست سفارشات پاک خواهد شد. دقت داشته باشید برای پیاده‌سازی لیست سفارشات نیز می‌بایست از مشابه آنچه برای پیک‌ها گفته شد، استفاده کنید.
فیلدهای زیر برای هر سفارش متصور است:
\شروع{فقرات}
\فقره شناسه سفارش
\فقره کد ملی پیک سفارش
\پایان{فقرات}


\زیرقسمت{رویه اجرا}
روند اجرای برنامه به شرح زیر است:

\begin{latin}
\begin{lstlisting}[]
    Welcome to bodo food, manager

    available couriers = 0
    total couriers = 0
    on-going orders = 0
    finised orders = 0

    1) Couriers
    2) Orders
    3) Quit
\end{lstlisting}
\end{latin}

\پاراگراف{}
در ابتدا منو فوق نمایش داده می‌شود. در این منو مدیر می‌تواند وارد یکی از زیر منوهای مربوط به پیک‌ها یا سفارشات شود.
در نظر داشته باشید که در ابتدای منو تعداد پیک‌های آماده، تعداد کل پیک‌ها، تعداد سفارشات در دست اقدام و تعداد کل سفارشات نمایش داده میشود.

\پاراگراف{}
در صورتی که کاربر در هر یک از مراحل عددی خارج از اعداد منو را انتخاب کند، می‌بایست هیچ اتفاقی رخ ندهد.

\پاراگراف{}
در صورتی که مدیر وارد منو پیک‌ها شود با منو زیر رو به رو خواهد شد:


\begin{latin}
\begin{lstlisting}[]
    Welcome to bodo food, manager

    available couriers = 0
    total couriers = 0
    on-going orders = 0
    finised orders = 0

    1) Add
    2) Remove
    3) Activate
    4) Deactivate
    5) Back
\end{lstlisting}
\end{latin}

\پاراگراف{}
برای این منو اعمال زیر را داریم:

\شروع{شمارش}
\فقره ایجاد: در این قسمت می‌بایست یک پیک فعال، با گرفتن کد ملی و شماره تماس ایجاد شود. در نظر داشته باشید همانطور که اشاره شد نیاز است در این قسمت صحت‌سنجی نیز صورت بپذیرد.
\فقره حذف:‌ در این قسمت می‌بایست یک کد ملی دریافت شده و اطلاعات پیک با این کد ملی در صورت وجود پاک شود.
\فقره فعال‌سازی: در این قسمت می‌بایست پیک به وضعیت فعال دربیاید. در نظر داشته باشید راننده فعال در گام انتخاب تصادفی برای یک سفارش انتخاب خواهد شد. ذکر این نکته نیز خالی از لطف نیست که در صورتی که پیک در حال انجام یک سفارش باشد هم با این گزینه وضعیتش به فعال تغییر خواهد کرد (اگر به خاطر داشته باشید پیک در صورت انتخاب شدن برای یک سفارش به وضعیت غیرفعال تغییر وضعیت پیدا می‌کرد).
\فقره غیرفعال‌سازی:‌ در این قسمت می‌بایست پیک به وضعیت غیرفعال دربیاید.
\فقره بازگشت: بازگشت به منو اصلی
\پایان{شمارش}

\پاراگراف{}
در صورتی که مدیر وارد منو سفارش‌ها شود با منو زیر رو به رو خواهد شد:

\begin{latin}
\begin{lstlisting}[]
    Welcome to bodo food, manager

    available couriers = 0
    total couriers = 0
    on-going orders = 0
    finised orders = 0

    1) Start
    2) Stop
    3) Back
\end{lstlisting}
\end{latin}

\پاراگراف{}
برای این منو اعمال زیر را داریم:

\شروع{شمارش}
\فقره آغاز: یک سفارش آغاز می‌شود، همانطور که بیان شد از بین پیک‌های فعال یک پیک تصادفی و یک شناسه تصادفی به سفارش تخصیص می‌یابند. در صورتی که پیک فعال وجود نداشته باشد برنامه می‌بایست با یک خطای مناسب به منو اصلی بازگردد.
\فقره پایان: یک سفارش با ورود شناسه آن که در زمان آغاز سفارش ساخته و به کاربر نیز نمایش داده شد، پایان می‌یابد. پس از خاتمه سفارش از لیست حذف شده و پیک آن به وضعیت فعال باز می‌گردد.
\فقره بازگشت: بازگشت به منو اصلی
\پایان{شمارش}

\پاراگراف{}
در اینجا مدیر سامانه قصد ایجاد یک پیک را دارد:

\begin{latin}
\begin{lstlisting}[]
    Welcome to bodo food, manager

    available couriers = 0
    total couriers = 0
    on-going orders = 0
    finised orders = 0

    ID: 0017784646
    Phone: 09121234567
\end{lstlisting}
\end{latin}

\پاراگراف{}
در صورت موفقیت آمیز بودن، منو پیک‌ها با مقادیر جدید برای \متن‌لاتین{available couriers} و \متن‌لاتین{total couriers} نمایش داده می‌شود.

\begin{latin}
\begin{lstlisting}[]
    Welcome to bodo food, manager

    available couriers = 1
    total couriers = 1
    on-going orders = 0
    finised orders = 0

    1) Add
    2) Remove
    3) Activate
    4) Deactivate
    5) Back
\end{lstlisting}
\end{latin}

\پاراگراف{}
در صورتی که اطلاعات وارد شده صحیح نباشند، یک پیام مناسب به کاربر نمایش داده شده و کاربر بعد از وارد کردن \متن‌لاتین{enter} وارد منو پیک‌ها می‌گردد.

\begin{latin}
\begin{lstlisting}[]
    Welcome to bodo food, manager

    available couriers = 0
    total couriers = 0
    on-going orders = 0
    finised orders = 0

    ID: 0017784646
    Phone: 091212345671

    Invalid phone number, please try again
\end{lstlisting}
\end{latin}

\پاراگراف{}
مدیر سیستم قصد حذف یک پیک با شناسه‌ی ۰۰۱۷۷۸۴۶۴۶ را دارد.

\begin{latin}
\begin{lstlisting}[]
    Welcome to bodo food, manager

    available couriers = 1
    total couriers = 1
    on-going orders = 0
    finised orders = 0

    ID: 0017784646

\end{lstlisting}
\end{latin}

در صورت موفقیت آمیز بودن، منو پیک‌ها با مقادیر جدید برای \متن‌لاتین{available couriers} و \متن‌لاتین{total couriers} نمایش داده می‌شود.

\begin{latin}
\begin{lstlisting}[]
    Welcome to bodo food, manager

    available couriers = 0
    total couriers = 0
    on-going orders = 0
    finised orders = 0

    1) Add
    2) Remove
    3) Activate
    4) Deactivate
    5) Back
\end{lstlisting}
\end{latin}

در صورتی که پیکی با شناسه‌ی داده شده وجود نداشته باشد، یک پیام مناسب به کاربر نمایش داده شده و کاربر بعد از وارد کردن \متن‌لاتین{enter} وارد منو پیک‌ها می‌گردد.

\begin{latin}
\begin{lstlisting}[]
    Welcome to bodo food, manager

    available couriers = 0
    total couriers = 0
    on-going orders = 0
    finised orders = 0

    ID: 0017784646

    There is no courier with given ID
\end{lstlisting}
\end{latin}

\زیرقسمت{نکات پیاده‌سازی}

\شروع{فقرات}
    \فقره تا جایی که می‌توانید برنامه‌ی خود را به توابع معنی‌دار بشکنید. بهتر است هر تابع یک کار معین را انجام دهد.
    \فقره برای پیاده‌سازی این پروژه تنها نیاز به آنچه \متن‌سیاه{تا انتهای توابع} بحث شده است دارید اما استفاده از تمام ویژگی‌های زبان سی آزاد می‌باشد.
    \فقره در پیاده‌سازی این پروژه فقط از \متن‌سیاه{زبان سی} استفاده کنید. در صورتی که برنامه شما برای اجرا نیاز به تنظیمات خاصی دارد یک راهنما برای اجرای برنامه بنویسید در فایل زیپتان همراه برنامه ارسال کنید.
    \فقره از آنجایی که این پروژه در قالب \متن‌سیاه{امتحان میانترم} می‌باشد از تغییر دادن صورت مساله یا انجام موارد خارج از موارد مطرح شده خودداری کنید.
    \فقره از هیچ کتابخانه یا تابع گرافیکی استفاده کرده و همه چیز را به صورت ساده در کنسول چاپ کنید. دقت داشته باشید که با توجه به ماهیت پروژه این موارد می‌تواند تصحیح پروژه را سختتر کند و نمره‌ی امتیازی برای شما دربر نخواهد داشت.
\پایان{فقرات}

\زیرقسمت{موارد امتیازی}

\شروع{فقرات}
    \فقره برای پیاده‌سازی آرایه‌های سفارشات و پیک‌ها زمانی که یک پیک یا سفارش حذف می‌گردد یک خانه خالی ایجاد می‌شود، اگر بخواهیم این خانه خالی را پر کنیم و به جای آن یک خانه در انتها آرایه را خالی کنیم، آیا روشی با یک جابجایی وجود دارد؟ در نظر داشته باشید روش پیشنهادی به اندازه فاصله میان خانه خالی و ابتدای آرایه نیاز به جابجایی دارد.
    \فقره برای سفارشات سود هم در نظر بگیرید و با اضافه کردن یک زیر منو با نام \متن‌لاتین{Results} و گزینه \متن‌لاتین{Profit} سود حاصل از سفارشات به پایان رسیده را اعلام کنید.
    \فقره برای هر راننده تعداد سفارشات انجام شده را ذخیره کرده و آن را در یک زیر منو با نام \متن‌لاتین{Results} و گزینه \متن‌لاتین{OrdersPerCouriers} بیاورید. در نظر داشته باشید این گزینه نیاز به دریافت کد ملی پیک خواهد داشت.
\پایان{فقرات}


\end{document}
