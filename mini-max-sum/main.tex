
\documentclass[../main.tex]{subfiles}

\begin{document}

\قسمت{جمع کمینه و بیشینه}

پنج عدد صحیح مثبت به شما داده شده است.
تابعی بنوسید که بیشترین و کمترین جمع چهار عدد، از این پنج عدد را مشخص کند و آن‌ها در یک خط چاپ کند.

\begin{latin}
\begin{minted}[]{c}
void mini_max_sum(int arr[]) {
  // write your code here
}
\end{minted}
\end{latin}

به طور مثال فرض کنید:

\begin{latin}
\begin{minted}[]{c}
int arr[] = {1, 3, 5, 7, 9};
\end{minted}
\end{latin}

در این صورت کمترین مجموع چهار عدد از بین این پنج عدد برابر:

\begin{latin}
\begin{minted}[]{c}
1 + 3 + 5 + 7 = 16
\end{minted}
\end{latin}

و بیشترین مجموع چهار عدد از بین این پنج عدد برابر:

\begin{latin}
\begin{minted}[]{c}
3 + 5 + 7 + 9 = 24
\end{minted}
\end{latin}


۲ نمره

\end{document}
